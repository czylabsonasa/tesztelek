\documentclass[12pt]{extarticle}
%\usepackage[utf8]{inputenc}
\usepackage{amsmath}
\usepackage{verbatim}
\usepackage{setspace}
\onehalfspace

\newcommand{\pn}{\par\noindent}

\begin{document}

\valami{0.1}{examexp}{
\vspace{0.2cm}
\pn
Egy diák $N$ tételből $K$-t tanult meg (ha ezeket húzza tuti átmegy), a többiről 
fogalma sincsen (bukta). A vizsgán $2$ tételt kell húzni, de csak az 
egyiket kell kidolgozni (azaz egyet visszadhat). Ha valamelyik kihúzott 
tételt megtanulta, akkor az egyszerűség kedvéért tegyük fel, hogy 
egyforma eséllyel kap egyet a $2,3,4,5$ a jegyekből. Mennyi a várható 
osztályzata ($J$)?
}

\valami{0.5}{Input}{
\pn $N$ és $K$ egy sorban:
\pn $N$ $K$
}

\valami{0.1}{Output}{
\pn A várható érdemjegy:
\pn $J$
}

\valami{0.5}{Megjegyzés}{
\pn $0\le K\le N\le 30$
}

\valami{0.5}{PéldaInput1}{
\verbatiminput{io/1.in}
}

\valami{0.1}{PéldaOutput1}{
\verbatiminput{io/1.out}
}

\valami{0.3}{PéldaInput2}{
\verbatiminput{io/2.in}
}

\valami{0.1}{PéldaOutput2}{
\verbatiminput{io/2.out}
}
\end{document}




